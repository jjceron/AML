\section{Reducción de Dimensionalidad}

% Topics
% Principal Component Analysis (PCA)
% Probabilistic PCA

La reducción de dimensionalidad es una técnica utilizada en el aprendizaje automático y la estadística para reducir el número de variables o características en un conjunto de datos, manteniendo al mismo tiempo la mayor cantidad posible de información relevante. Esta técnica es especialmente útil cuando se trabaja con conjuntos de datos de alta dimensionalidad, donde muchas características pueden ser redundantes o irrelevantes, lo que puede dificultar el análisis y la interpretación de los datos.

\subsection{Análisis de Componentes Principales (PCA)}

%% ...

\subsection{Probabilistic PCA}

PPCA is a linear-Gaussian latent variable model in which all marginal and conditional distributions are Gaussian. Latent variable prior:

\[p(\boldsymbol{z})=\mathcal{N}(\boldsymbol{z}|\boldsymbol{0},\boldsymbol{I})\]

Conditional distribution of the observed variable \(x\in\mathbb{R}^D\) given the latent variable \(z\in\mathbb{R}^d\):

\[p(\boldsymbol{x}|\boldsymbol{z})=\mathcal{N}(\boldsymbol{x}|\boldsymbol{Wz}+\boldsymbol{\mu},\sigma^2\boldsymbol{I})\]